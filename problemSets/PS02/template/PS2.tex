\documentclass[12pt,letterpaper]{article}
\usepackage{graphicx,textcomp}
\usepackage{natbib}
\usepackage{setspace}
\usepackage{fullpage}
\usepackage{color}
\usepackage[reqno]{amsmath}
\usepackage{amsthm}
\usepackage{fancyvrb}
\usepackage{amssymb,enumerate}
\usepackage[all]{xy}
\usepackage{endnotes}
\usepackage{lscape}
\newtheorem{com}{Comment}
\usepackage{float}
\usepackage{hyperref}
\newtheorem{lem} {Lemma}
\newtheorem{prop}{Proposition}
\newtheorem{thm}{Theorem}
\newtheorem{defn}{Definition}
\newtheorem{cor}{Corollary}
\newtheorem{obs}{Observation}
\usepackage[compact]{titlesec}
\usepackage{dcolumn}
\usepackage{tikz}
\usetikzlibrary{arrows}
\usepackage{multirow}
\usepackage{xcolor}
\newcolumntype{.}{D{.}{.}{-1}}
\newcolumntype{d}[1]{D{.}{.}{#1}}
\definecolor{light-gray}{gray}{0.65}
\usepackage{url}
\usepackage{listings}
\usepackage{color}

\definecolor{codegreen}{rgb}{0,0.6,0}
\definecolor{codegray}{rgb}{0.5,0.5,0.5}
\definecolor{codepurple}{rgb}{0.58,0,0.82}
\definecolor{backcolour}{rgb}{0.95,0.95,0.92}

\lstdefinestyle{mystyle}{
	backgroundcolor=\color{backcolour},   
	commentstyle=\color{codegreen},
	keywordstyle=\color{magenta},
	numberstyle=\tiny\color{codegray},
	stringstyle=\color{codepurple},
	basicstyle=\footnotesize,
	breakatwhitespace=false,         
	breaklines=true,                 
	captionpos=b,                    
	keepspaces=true,                 
	numbers=left,                    
	numbersep=5pt,                  
	showspaces=false,                
	showstringspaces=false,
	showtabs=false,                  
	tabsize=2
}
\lstset{style=mystyle}
\newcommand{\Sref}[1]{Section~\ref{#1}}
\newtheorem{hyp}{Hypothesis}

\title{Problem Set 2}
\date{Due: February 19, 2023}
\author{Applied Stats II}


\begin{document}
	\maketitle
	\section*{Instructions}
	\begin{itemize}
		\item Please show your work! You may lose points by simply writing in the answer. If the problem requires you to execute commands in \texttt{R}, please include the code you used to get your answers. Please also include the \texttt{.R} file that contains your code. If you are not sure if work needs to be shown for a particular problem, please ask.
		\item Your homework should be submitted electronically on GitHub in \texttt{.pdf} form.
		\item This problem set is due before 23:59 on Sunday February 19, 2023. No late assignments will be accepted.
	%	\item Total available points for this homework is 80.
	\end{itemize}

	
	%	\vspace{.25cm}
	
%\noindent In this problem set, you will run several regressions and create an add variable plot (see the lecture slides) in \texttt{R} using the \texttt{incumbents\_subset.csv} dataset. Include all of your code.

	\vspace{.25cm}
%\section*{Question 1} %(20 points)}
%\vspace{.25cm}
\noindent We're interested in what types of international environmental agreements or policies people support (\href{https://www.pnas.org/content/110/34/13763}{Bechtel and Scheve 2013)}. So, we asked 8,500 individuals whether they support a given policy, and for each participant, we vary the (1) number of countries that participate in the international agreement and (2) sanctions for not following the agreement. \\

\noindent Load in the data labeled \texttt{climateSupport.csv} on GitHub, which contains an observational study of 8,500 observations.

\begin{itemize}
	\item
	Response variable: 
	\begin{itemize}
		\item \texttt{choice}: 1 if the individual agreed with the policy; 0 if the individual did not support the policy
	\end{itemize}
	\item
	Explanatory variables: 
	\begin{itemize}
		\item
		\texttt{countries}: Number of participating countries [20 of 192; 80 of 192; 160 of 192]
		\item
		\texttt{sanctions}: Sanctions for missing emission reduction targets [None, 5\%, 15\%, and 20\% of the monthly household costs given 2\% GDP growth]
		
	\end{itemize}
	
\end{itemize}

\newpage
\noindent Please answer the following questions:

\begin{enumerate}
	\item
	Remember, we are interested in predicting the likelihood of an individual supporting a policy based on the number of countries participating and the possible sanctions for non-compliance.
	\begin{enumerate}
		\item [] Fit an additive model. Provide the summary output, the global null hypothesis, and $p$-value. Please describe the results and provide a conclusion.
		%\item
		%How many iterations did it take to find the maximum likelihood estimates?
	\end{enumerate}



 \title{Answer Question 1:} 

\begin{verbatim}
    
model <- glm(choice ~ countries 
             + sanctions,family=binomial(link='logit'),data=climateSupport)
summary(model)
\end{verbatim}

 \title{Model Output in R:} 

\begin{verbatim}
    
> model <- glm(choice ~ countries 
+              + sanctions,family=binomial(link='logit'),data=climateSupport)
> summary(model)

Call:
glm(formula = choice ~ countries + sanctions, family = binomial(link = "logit"), 
    data = climateSupport)

Deviance Residuals: 
    Min       1Q   Median       3Q      Max  
-1.4259  -1.1480  -0.9444   1.1505   1.4298  

Coefficients:
             Estimate Std. Error z value Pr(>|z|)    
(Intercept) -0.005665   0.021971  -0.258 0.796517    
countries.L  0.458452   0.038101  12.033  < 2e-16 ***
countries.Q -0.009950   0.038056  -0.261 0.793741    
sanctions.L -0.276332   0.043925  -6.291 3.15e-10 ***
sanctions.Q -0.181086   0.043963  -4.119 3.80e-05 ***
sanctions.C  0.150207   0.043992   3.414 0.000639 ***
---
Signif. codes:  0 ‘***’ 0.001 ‘**’ 0.01 ‘*’ 0.05 ‘.’ 0.1 ‘ ’ 1

(Dispersion parameter for binomial family taken to be 1)

    Null deviance: 11783  on 8499  degrees of freedom
Residual deviance: 11568  on 8494  degrees of freedom
AIC: 11580

Number of Fisher Scoring iterations: 4

\end{verbatim}

The null hypothesis states that the explanatory variables of countries
and sanction levels do not affect the log odds of someone supporting a policy.


The intercept value of (-.006) in the output indicates the log odds of individuals supporting a policy when the explanatory variables are 
at reference level, 0.

\begin{verbatim}
    
(exp(-.0057)) / (1 + exp(-.0057))

\end{verbatim}
The probability that someone will support a policy is .4986.


We can see that moving to the highest levels of countries involved in a policy is not
a statistically significant contribution to the model with a p value of .794.
The other explanatory variables are significantly associated with the probability of supporting a policy.


To get the model chi square value we subtract the residual
#deviance score from the null deviance score

\begin{verbatim}
  
X2 <- 111783 - 11568
X2

100215

\end{verbatim}

There are p = 5 predictor variables degrees of freedom.

The p value here is less than .5 and therefore we reject the null hypothesis.This indicates that at least one of the explanatory variables are significant in the model at predicting policy support.

	\end{itemize}

	\section*{Instructions}
	
	\item
	If any of the explanatory variables are significant in this model, then:
	\begin{enumerate}
		\item
		For the policy in which nearly all countries participate [160 of 192], how does increasing sanctions from 5\% to 15\% change the odds that an individual will support the policy? (Interpretation of a coefficient)
%		\item
%		For the policy in which very few countries participate [20 of 192], how does increasing sanctions from 5\% to 15\% change the odds that an individual will support the policy? (Interpretation of a coefficient)

 \title{Answer Question 2 - Part 1:} 
 
\begin{verbatim}
levels(climateSupport$sanctions)
    
exp(-.01811)
\end{verbatim}

The coefficient is equal to (-.01811) and the p value is <.05, this indicates 
that this change in sanction levels does influence policy support. The negative value
indicates that increaseing the sanction
would decrease the likelihood of someone supporting the policy. 

\begin{verbatim}
In a policy in which nearly all countries participate, the odds of someone
supporting a policy with 15% sanctions is .9821 times less than the odds of 
someone supporting a policy with 5% sanctions. 
 
1-.9821 = 17.9%
Policies with 15% sanctions are associated with a 17.9% reduction in 
policy support. 

\end{verbatim}

		\item
		What is the estimated probability that an individual will support a policy if there are 80 of 192 countries participating with no sanctions? 
  
 \title{Answer Question 2 - Part 2:} 

\begin{verbatim}
(-.0057)+(-0.0010)

ep =(exp(-.0057)+(-0.0010))/(1 + exp((-.0057)+(-0.0010)))
ep

\end{verbatim}

The estimated probability that an individual will support a policy
if 80/192 countries are participating and there are no sanctions is .4983.


		\item
		Would the answers to 2a and 2b potentially change if we included the interaction term in this model? Why? 
		\begin{itemize}
			\item Perform a test to see if including an interaction is appropriate.\end{itemize}

  
\title{Answer Question 2 - Part 3:} 

If there is an interaction term in this model then the distributions of the 
variables may crossover eachother. This would result in different log likelihood 
and probability values in relation to policy support.

To see if an interaction model is needed here I have performed a likelihood ratio test. The null hypothesis states that the second model, the interactive model, is a better fit for the data.

\end{itemize}

\begin{verbatim}
model2 <- glm(choice ~ countries 
             * sanctions,family=binomial(link='logit'),data=climateSupport)
summary(model2)
install.packages('lmtest')
library(lmtest)

lrtest(model, model2)

Output:

Likelihood ratio test

Model 1: choice ~ countries + sanctions
Model 2: choice ~ countries * sanctions
  #Df  LogLik Df  Chisq Pr(>Chisq)
1   6 -5784.1                     
2  12 -5781.0  6 6.2928     0.3912

\end{verbatim}

We can see from the output that the p value is not less than .05 so we fail
to reject the null hypothesis. The interactive predictor variables
do not offer a significant improvement from the additive model. 

	\end{enumerate}
	\end{enumerate}


\end{document}
